Flux Through Sloping Side of a Cone

Problem:
A cone rests on a table with its face horizontal.
A uniform electric field of magnitude 4550 N/C points vertically upward.
How much electric flux passes through the sloping side surface area of the cone?

From the figure:
Diameter of the circular face = 4.22 cm
So radius:
$$r=\frac{4.22}{2}\text{ cm}=2.11\text{ cm}=0.0211\text{ m}$$

Given:
$$E=4550\text{ N/C}$$

------------------------------------------------------------
STEP 1 — Basic equation (Gauss’s law concept)

For any closed surface:
$$\Phi_{\text{total}}=\frac{Q_{\text{encl}}}{\epsilon_0}$$

Here there is NO charge enclosed by the cone + its base:
$$Q_{\text{encl}}=0 \Rightarrow \Phi_{\text{total}}=0$$

------------------------------------------------------------
STEP 2 — Close the surface

Make a closed surface by adding the flat circular base.

Then:
$$\Phi_{\text{side}}+\Phi_{\text{base}}=0$$

So:
$$\Phi_{\text{side}}=-\Phi_{\text{base}}$$

------------------------------------------------------------
STEP 3 — Compute base flux

Base area:
$$A=\pi r^2$$

The electric field points upward.
The outward normal of the base points downward.
So angle = 180° and:
$$\Phi_{\text{base}}=EA\cos(180^\circ)=-EA$$

------------------------------------------------------------
STEP 4 — Solve for side flux

$$\Phi_{\text{side}}=-(-EA)=EA$$

So:
$$\Phi_{\text{side}}=E\pi r^2$$

Substitute:
$$\Phi_{\text{side}}=4550\cdot\pi\cdot(0.0211)^2$$

$$(0.0211)^2=4.45\times10^{-4}$$

$$\Phi_{\text{side}}=4550\cdot\pi\cdot4.45\times10^{-4}=6.36$$

Final (3 sig figs):
$$\boxed{\Phi_{\text{side}}=6.36\ \text{N·m}^2/\text{C}}$$
